\documentclass{article}
\usepackage{amsmath}
\usepackage{amsfonts}
\usepackage{amssymb}
\begin{document}
\title{Mathematical Expression Compatibility Testing}
\maketitle

\section{Basic Arithmetic and Algebra}
Simple superscripts and subscripts:
\begin{equation}
x^2 + y^2 = z^2
\end{equation}

Multiple indices:
\begin{equation}
a_1, a_2, \ldots, a_n
\end{equation}

Complex superscripts:
\begin{equation}
x^{2n+1} + y_{i,j}^{(k)}
\end{equation}

\section{Fractions}
Simple fraction:
\begin{equation}
\frac{1}{2} + \frac{3}{4} = \frac{5}{4}
\end{equation}

Complex fraction:
\begin{equation}
\frac{a+b}{c-d} = \frac{2(c-d) + (a+b)}{2(c-d)}
\end{equation}

Nested fractions:
\begin{equation}
\frac{1}{1+\frac{1}{x}}
\end{equation}

\section{Roots and Radicals}
Square roots:
\begin{equation}
\sqrt{2}, \sqrt{x^2 + y^2}
\end{equation}

nth roots:
\begin{equation}
\sqrt[3]{8}, \sqrt[n]{x^n}
\end{equation}

Nested roots:
\begin{equation}
\sqrt{2 + \sqrt{3}}
\end{equation}

\section{Summations and Products}
Basic summation:
\begin{equation}
\sum_{i=1}^{n} i = \frac{n(n+1)}{2}
\end{equation}

Product notation:
\begin{equation}
\prod_{k=1}^{n} (1 + x_k)
\end{equation}

\section{Integrals and Limits}
Basic integral:
\begin{equation}
\int_0^1 x^2 dx = \frac{1}{3}
\end{equation}

Limit:
\begin{equation}
\lim_{x \to 0} \frac{\sin x}{x} = 1
\end{equation}

\section{Matrices}
2x2 Matrix:
\begin{equation}
\begin{pmatrix}
a & b \\
c & d
\end{pmatrix}
\end{equation}

Matrix multiplication:
\begin{equation}
\begin{pmatrix}
a & b \\
c & d
\end{pmatrix}
\begin{pmatrix}
x \\
y
\end{pmatrix}
=
\begin{pmatrix}
ax + by \\
cx + dy
\end{pmatrix}
\end{equation}

\section{Special Symbols}
Number sets:
\begin{equation}
\mathbb{N}, \mathbb{Z}, \mathbb{Q}, \mathbb{R}, \mathbb{C}
\end{equation}

Logic symbols:
\begin{equation}
\forall x \in \mathbb{R}, \exists y : x + y = 0
\end{equation}

\section{Complex Cases}
Binomial coefficient:
\begin{equation}
\binom{n}{k} = \frac{n!}{k!(n-k)!}
\end{equation}

Cases environment:
\begin{equation}
f(x) = \begin{cases}
x^2 & \text{if } x \geq 0 \\
-x^2 & \text{if } x < 0
\end{cases}
\end{equation}

\end{document}